% -*-coding: utf-8 -*-

\zp{Ztráty a nálezy}{Hlavolam}

<Emi> <D> <Ami> <Emi> <H>

\zs
<Emi>Kolikrát jsem minul tuhletu výlohu s <D>panáčkem od sazí,

<Ami>nikdy jsem nepřišel na to, co znamená \uv{<Emi>ztráty a nále<H>zy},

<Emi>když dneska pomyslím na ta dvě slovíčka, <D>trochu mě zamrazí,

<Ami>vždyť i v mém životě hrají hlavní roli <Emi>ztráty a nále<H>zy.
\ks

\zr
A tak ať <Emi>hledám či nehle<C>dám, staré <D>ztrácím, nové nalé<G>zám,

něco <Emi>mám a pak ne<C>mám zase <H>nic,

říka<Emi>jí, že je jen <C>klam úsměv <D>čarokrásných <G>dam,

stále <Emi>ztrácím a nalé<C>zám, někdy <H>málo a někdy víc.
\kr

\zs
Najít můžeš černé peklo a ztratit zas nejmodřejší nebe,

vždycky však pamatuj na ztráty a nálezy, nechoď hledat sebe,

až jednou uvidíš, že nemáš vůbec nic, neztrácej naději,

vždyť kdo hledá, najde, a tak znovu hledat začínej raději.
\ks

\zr
Jednou ten koloběh poznáš, že staré ztrácíš, nové nalézáš,

něco máš a hned nemáš zase nic,

proto včerejšek dnes smaž, málo vezmeš a snad více dáš,

stále ztrácíš a nalézáš, někdy málo a někdy víc.
\kr

Co dál k tomu <Emi>říct... <D> <Ami> <Emi> <H> <Emi>

\kp
